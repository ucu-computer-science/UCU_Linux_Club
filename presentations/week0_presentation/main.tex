\documentclass[usenames,dvipsnames,10pt,aspectratio=169]{beamer} 
% Add option 'aspectratio=169' for 16:9 widescreen 
% Add option  'handout' to ignore animations
% If you have a smaller amount of text, feel free to also try '11pt'! / Jesper

\usepackage[utf8]{inputenc}
\usepackage{verbatim}
\usepackage{minted}
\usepackage{graphicx}
\usepackage{wrapfig}
\usepackage{geometry}
\usepackage[document]{ragged2e}
\usetheme{umu}
\usemintedstyle{monokai}

\usepackage{hyperref}
\hypersetup{
    colorlinks=true,
    linkcolor=ucugreyish,
    filecolor=ucured,
    urlcolor=ucublue,
}
\urlstyle{same}

%%% Bibliography
\usepackage[style=authoryear,backend=biber]{biblatex}
\addbibresource{bibliography.bib}

\DeclareNameAlias{author}{given-family}

%%% Suppress biblatex annoying warning
\usepackage{silence}
\WarningFilter{biblatex}{Patching footnotes failed}

%%% Some useful commands
% pdf-friendly newline in links
\newcommand{\pdfnewline}{\texorpdfstring{\newline}{ }} 
% Fill the vertical space in a slide (to put text at the bottom)
\newcommand{\framefill}{\vskip0pt plus 1filll}

%%% Enter additional packages below (or above, I can't stop you)! / Jesper
\renewcommand{\proofname}{\sffamily{Proof}}

%%%%%%%%%%%%%%%%%%%%%%%%%%%%%%%%%%%%%%%%%%%%%%%%%%%%%%%%%%%%%%%%%%%%%%%%%%%%%%%%%%%%%
\title[Linux club]{Linux Club}
\date[\today]{\small\today}
\author[Morhunenko Mykola]{Morhunenko Mykola}
\institute{APPS@UCU}

\begin{document}

\begin{frame}
\titlepage
\end{frame}

\begin{frame}{\contentsname}
\setbeamercolor{background canvas}{bg=ucugrey}
\tableofcontents
\end{frame}

\framepic{graphics/6.jpg}{
 \centering
 \MakeUppercase\textcolor{ucuwhite}{Plans for semester}
 \vskip 0.5cm
}

\section{A short history of systems programming}

\framesplitc{The origins of C}{graphics/1.jpg}{The C programming language appeared during Unix development in 1972.\\

Since it was created for a specific purpose % an operating system
and a specific computer, %(PDP-11) 
on the one hand it adapted to the
needs of the programmers, and on the 
other it adopted a large
amount of somewhat unique and unpopular 
ideas and concepts.}

\begin{frame}{Possible solutions}
\large
C++ is born to help address some of these problems,\\ 
introduces ‘zero cost’ abstractions, aimed at providing\\
a nice interface for the programmer to use which\\
compiles down to an almost ideal machine code.
\vspace{0.5cm}

Still has the old instruments, hangs on to C's machine\\
model %(the fucking BACKWARDS COMPATIBILITY),
and tries to encourage using the new\\ 
modern safe concepts, 
\href{https://alexgaynor.net/2019/apr/21/modern-c++-wont-save-us/}
{which are not ideal either}.


\end{frame}

\begin{frame}{Modern ideas}

\large
In the meantime, languages like Java, Ruby and\\
Python start sprawling up, presenting another\\
model of growth - they are garbage-collected\\
and are able to present even more complex\\
abstractions (at the expense of the speed).\\

\vspace{0.5cm}

Go and others try to tackle C's speed and\\
low-levelness, 
\href{https://cowlark.com/2009-11-15-go/}{unsuccessfully}.

\end{frame}

%%%%%%%%%%%%%%%%%%%%%%%%%%%%
\section{Bonus Commands}


\framecard{A SECTION\\TITLE}
\framecard[ucublue]{A SECTION TITLE\\WITH A CUSTOM COLOR}


\framepic{graphics/1.jpg}{
	\framefill
    \textcolor{white}{Luke,\\I am your supervisor}
    \vskip 0.5cm
}

\framepic{graphics/1}{
	\vfill
    \begin{center}
    \textcolor{ucured}{\textbf{Right-aligned text with\\Semi-transparent background}}
    \end{center}	
}

\begin{frame}[t,fragile,allowframebreaks]
\frametitle{Other bonus commands}

We provide two other bonus commands:
\begin{description}
\item[\texttt{pdfnewline}] you can use \texttt{\textbackslash pdfnewline} to avoid the annoying \texttt{hyperref} related warnings when using newlines in the document's title, author, etc. For example, in this presentation the author is defined as:
\begin{verbatim}
\author[Luke Skywalker]{
  Luke Skywalker, Ph.D.
  \pdfnewline
  \texttt{luke.skywalker@uniud.it}
}
\end{verbatim}
\item[\texttt{marker}] you can use \texttt{\textbackslash marker} to highlight some text. The default color is \marker{orange}, but you can also \marker[ucublue]{use a custom color}. For example:
\begin{verbatim}
\marker{Default color}
\marker[ucublue]{Custom Color}
\end{verbatim}
\item[\texttt{framefill}] you can use \texttt{\textbackslash framefill} to put the text at the bottom of a slide by filling all the vertical space.
\end{description}

\end{frame}


\end{document}